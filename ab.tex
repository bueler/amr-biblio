\documentclass[11pt]{article}

\usepackage[top=1in, bottom=1in, left=1.25in, right=1.25in]{geometry}

\usepackage{amssymb,amsmath,amsthm}
\usepackage{enumerate,palatino,bm}

\usepackage[final]{graphicx}

\usepackage[colorlinks=true,citecolor=blue,linkcolor=red,urlcolor=blue]{hyperref}

\newtheorem{theorem}{Theorem}
\newtheorem{conjecture}{Conjecture}

\newcommand{\RR}{\ensuremath{\mathbb R}}
\newcommand{\ZZ}{\ensuremath{\mathbb Z}}

\newcommand{\bn}{\ensuremath{\mathbf{n}}}
\newcommand{\bu}{\ensuremath{\mathbf{u}}}

\newcommand{\bzero}{\ensuremath{\bm{0}}}

\newcommand{\cK}{\ensuremath{\mathcal{K}}}
\newcommand{\cX}{\ensuremath{\mathcal{X}}}

\newcommand{\grad}{\ensuremath{\nabla}}
\newcommand{\eps}{\ensuremath{\epsilon}}


\title{Bibliography: adaptive mesh refinement \\ and \emph{a posteriori} estimators}
\author{Ed Bueler}

\begin{document}
\maketitle

\section{Introduction}

This is an annotated bibliography, and opinionated.

The literature of adaptive mesh refinement (AMR) for finite element (FE; \cite{ElmanSilvesterWathen2014}) methods is quite big in 2025.  I have missed too many citations in this space, even for the restricted literature of AMR for variational inequalities (VIs; \cite{KinderlehrerStampacchia1980}) and obstacle problems.  In working on \cite{FochesattoBueler2025}, and the associated Python/Firedrake library  \href{https://github.com/StefanoFochesatto/viamr}{\texttt{viamr}}, we evidently need a place to keep track of, and provide context for, these references.

To set notation, used below in commentary, note that the \emph{classical obstacle problem} is either the entire purpose of articles cited below, or is at least a primary example.  In strong (\emph{complementarity}) form,
\begin{subequations} \label{eq:ncp}
\begin{align}
  -\nabla^2 u - f &\geq 0 \\
  u - \psi &\geq 0 \\
  (-\nabla^2u - f)(u - \psi) &= 0
\end{align}
\end{subequations}
This problem is over a domain $\Omega \subset \RR^2$, which we will assume (for here) is polygonal.

When turned into a weak form it is an inequality over an \emph{admissible set}
\begin{equation}
\cK = \{u \in \cX \,:\, u \ge \psi \text{ and } u|_{\partial \Omega} = g\}
\end{equation}
which is a closed and convex set of the Hilbert space $\cX = H^1(\Omega)$, namely
\begin{equation}
\int_\Omega \nabla u \cdot \nabla(v - u) \ge \int_\Omega f(v - u) \quad \text{ for all } v \in \cK. \label{eq:vi}
\end{equation}

The operator $F \in \cX'$ in \eqref{eq:vi}, namely
\begin{equation}
F(v)[w] = \int_\Omega \nabla v \cdot \nabla w - \int_\Omega f w,
\end{equation}
is the \emph{residual} or the \emph{lagrange multiplier}, depending on the author.  The last usage makes sense if we write complementarity problem \eqref{eq:ncp} as
\begin{subequations} \label{eq:ncplagrange}
\begin{align}
  -\nabla^2 u &= f + \lambda \\
  \lambda \geq 0, \quad u - \psi \geq 0, \quad \lambda (u - \psi) &= 0
\end{align}
\end{subequations}

\newcommand{\nm}[1]{\item \,{\large \textbf{#1}} \, \cite{#1} \,}

\section{AMR books}

\begin{itemize}
\nm{AinsworthOden2000} foo
\nm{BangerthRannacher2003}
\nm{Demkowicz2007}
\nm{Suttmeier2008}
\end{itemize}

\section{AMR for variational inequalities and obstacle problems}

\begin{itemize}
\nm{AinsworthOdenLee1993}
\nm{BraessCarstensenHoppe2007}  This is a central paper for AMR on obstacle problems with affine obstacles.  The estimator ignores the active/inactive sets and just uses edge residuals.  I think this method breaks on rough data ($f$ or $\psi$).
\end{itemize}

\section{AMR for finite elements generally}

\begin{itemize}
\nm{BabuskaRheinboldt1979}
\nm{BeckerRannacher2001}
\nm{PlazaCarey2000}
\end{itemize}

{\small
\bibliographystyle{siamplain}
\bibliography{ab}
}
\end{document}
