\documentclass[11pt]{article}

\usepackage[top=1in, bottom=1in, left=1.25in, right=1.25in]{geometry}

\usepackage{amssymb,amsmath,amsthm}
\usepackage{enumerate,palatino,bm}

\usepackage[final]{graphicx}
\usepackage[dvipsnames]{xcolor}

\usepackage[colorlinks=true,citecolor=blue,linkcolor=red,urlcolor=blue]{hyperref}

\newtheorem{theorem}{Theorem}
\newtheorem{conjecture}{Conjecture}

\newcommand{\RR}{\ensuremath{\mathbb R}}
\newcommand{\ZZ}{\ensuremath{\mathbb Z}}

\newcommand{\bn}{\ensuremath{\mathbf{n}}}
\newcommand{\bu}{\ensuremath{\mathbf{u}}}

\newcommand{\bzero}{\ensuremath{\bm{0}}}

\newcommand{\cK}{\ensuremath{\mathcal{K}}}
\newcommand{\cX}{\ensuremath{\mathcal{X}}}

\newcommand{\grad}{\ensuremath{\nabla}}
\newcommand{\eps}{\ensuremath{\epsilon}}


\title{Bibliography: adaptive mesh refinement \\ and \emph{a posteriori} estimators}
\author{Ed Bueler}

\begin{document}
\maketitle

\section{Introduction}

This is an annotated bibliography, and opinionated.  It covers adaptive mesh refinement and \emph{a posteriori} estimators, especially for obstacle problems and variational inequalities (VIs; \cite{KinderlehrerStampacchia1980}).

The literature of adaptive mesh refinement (AMR) for finite element (FE; \cite{ElmanSilvesterWathen2014}) methods is quite large now in 2025.  I have missed too many citations in this space, even for the restricted literature of AMR for variational inequalities and obstacle problems.  In working on \cite{FochesattoBueler2025}, and the associated Python/Firedrake library  \href{https://github.com/StefanoFochesatto/viamr}{\texttt{viamr}}, we evidently need a place to keep track of, and provide context for, these references.

To establish notation, used below in commentary/annotations, the \emph{classical obstacle problem}, in strong (\emph{complementarity}) form, is
\begin{subequations} \label{eq:ncp:classical}
\begin{align}
  -\nabla^2 u - f &\geq 0 \\
  u - \psi &\geq 0 \\
  (-\nabla^2u - f)(u - \psi) &= 0
\end{align}
\end{subequations}
This problem is over a domain $\Omega \subset \RR^2$, which we will assume (for here) is polygonal.  This problem is either the entire purpose of articles cited below, or is at least a primary example.

When turned into a weak form it is a \emph{variational inequality} (VI) over an \emph{admissible set}
\begin{equation}
\cK = \{u \in \cX \,:\, u \ge \psi \text{ on $\Omega$, and } u|_{\partial \Omega} = g\},
\end{equation}
where $\psi \in \cX$ is the \emph{obstacle}.  Note that $\cK$ is a closed and convex set of the Hilbert space $\cX = H^1(\Omega)$, and it is nonempty if $\psi_{\partial\Omega} \le g$.  One can systematically incorporate the fixed boundary condition $u=g$ into the definition of $\cK$, as done here.  The classical obstacle problem is then to find $u\in\cK$ so that
\begin{equation}
\int_\Omega \nabla u \cdot \nabla(v - u) \ge \int_\Omega f(v - u) \quad \text{ for all } v \in \cK. \label{eq:vi:classical}
\end{equation}

The operator $F \in \cX'$ in \eqref{eq:vi:classical}, namely
\begin{equation}
F(v)[w] = \int_\Omega \nabla v \cdot \nabla w - \int_\Omega f w,
\end{equation}
For any $F \in \cX'$ we have the general VI form:
\begin{equation}
F(u)[v-u] \ge 0 \quad \text{ for all } v \in \cK. \label{eq:vi}
\end{equation}
Here $F(u)$ is the \emph{residual} or the \emph{Lagrange multiplier}, depending on the author.  Problem \eqref{eq:vi} says that $F(u)[v-u]\ge 0$ for all $v\in\cK$, so that $F(u)$ is (heuristically) as close to zero as it can be within the constrained set $\cK$, and nonnegative.  The residual $F(u)$ may be denoted $\sigma$ in the literature \cite[for example]{NochettoSiebertVeeser2003}.

The ``Lagrange multiplier'' usage makes sense if we write the complementarity problem \eqref{eq:ncp:classical} for the classical obstacle problem as
\begin{equation} \label{eq:ncplagrange}
  -\nabla^2 u = f + \lambda, \quad \lambda \geq 0, \quad u - \psi \geq 0, \quad \lambda (u - \psi) = 0.
\end{equation}
These are the KKT conditions of the optimization problem
\begin{equation}
u \longleftarrow \min_{v\in\cK} J(v) = \int_\Omega \frac{1}{2} |\grad v|^2 - fv.\label{eq:min}
\end{equation}
However, general VIs may not correspond to an optimization problem; the glacier problem in \cite{FochesattoBueler2025} is an example.

From a solution of \eqref{eq:vi} we may identify the \emph{active} (\emph{contact}, \emph{coincidence}) and \emph{inactive} (\emph{non-contact, non-coincidence}) sets:
\begin{equation}
  I_u = \{x \in \Omega \,:\, u(x) > \psi(x)\}, \quad A_u = \Omega \setminus I_u. \label{eq:sets}
\end{equation}
These sets define the free boundary:
\begin{equation}
  \Gamma_u = \Omega \cap \partial I_u. \label{eq:freebdry}
\end{equation}
Sometimes the literature emphasizes that $F$ is a positive Radon measure supported in the active set $A_u$ \cite{KinderlehrerStampacchia1980}.

The FE solution solves the same VI as \eqref{eq:vi} over a finite-dimensional space $\cX_h\subset\cX$:
\begin{equation}
F_h(u_h)[v_h-u_h] \ge 0 \quad \text{ for all } v_h \in \cK_h. \label{eq:fe:vi}
\end{equation}

In considering the FE VI \eqref{eq:fe:vi}, the precise definition of the FE admissible set $\cK_h$ is one thing to pay attention to.  Certain references give analysis which is only useful in the (geometrically) \emph{conforming} case $\cK_h\subset \cK$, but generally $\cK_h$ is not contained in $\cK$ \cite{FochesattoBueler2025}.  Most of the literature is only for $P_1$ elements, and some of the literature is special to affine obstacles, that is, for which $\grad^2 \psi=0$, a conforming case.

Likewise, in some cases $F_h=F|_{\cX_h}$, but generally $F_h$ is only an approximation to $F$, e.g.~because of quadrature.  The discrete residual $F(u_h)$ is also a measure, one which is typically singular with respect to Lebesgue measure along the element edges.

Essentially all of the literature addresses AMR for \emph{shape-regular triangulations} \cite{BangerthRannacher2003} of a polygonal domain $\Omega$.


\section{AMR for variational inequalities and obstacle problems}

The {\color{BrickRed} red items} in the next three sections are, possibly, the most important ones.

\newcommand{\nm}[1]{\item \,{\large \textbf{#1}} \, \cite{#1} \,}
\newcommand{\rnm}[1]{\item \,{\large {\color{BrickRed} \textbf{#1}}} \, \cite{#1} \,}
\newcommand{\res}[1]{\,\emph{#1.}\,}

\begin{itemize}
\nm{AinsworthOdenLee1993} FIXME
\rnm{BraessCarstensenHoppe2007}  \res{classical, affine}  This is a central paper for AMR on obstacle problems with affine obstacles.  The estimator ignores the active/inactive sets and just uses edge residuals.  I think this method breaks on rough data ($f$ or $\psi$).  Refinement requires interior nodes in each triangle, which is a kind of over-refinement, and an (un-algorithmed) reduction of oscillation of $f$.  The main point is that discrete local efficiency leads to an energy reduction theorem.
\nm{BraessCarstensenHoppe2009} FIXME
\nm{CarstensenHu2015} FIXME
\nm{CarstensenMerdon2013} FIXME
\nm{ChenNochetto2000} FIXME
\rnm{NochettoSiebertVeeser2003}  \res{classical, general obstacle}  FIXME
\nm{Veeser2001} FIXME
\nm{WeissWohlmuth2010} FIXME
\end{itemize}

\section{AMR for finite elements generally}

\begin{itemize}
\nm{BabuskaRheinboldt1979} FIXME
\nm{BeckerRannacher2001} FIXME
\nm{Dorfler1996} FIXME
\nm{PlazaCarey2000} FIXME
\end{itemize}

\section{AMR books}

\begin{itemize}
\nm{AinsworthOden2000} FIXME
\nm{BangerthRannacher2003} FIXME
\nm{Demkowicz2007} FIXME
\rnm{Suttmeier2008} FIXME
\end{itemize}

\section{Journals for AMR papers}

\subsection*{Implementation-focussed journals}

\begin{itemize}
\item \href{https://www.degruyterbrill.com/journal/key/cmam/html?srsltid=AfmBOoo8WGJxRllInTVAWwmPnu4jpE_REhB_ohFjk93UKvPwxGXhQeir}{Computational Methods in Applied Mathematics}  Carstensen is Editor-in-Chief
\item \href{https://link.springer.com/journal/44207}{Computational Science and Engineering}
\item \href{https://www.sciencedirect.com/journal/computer-methods-in-applied-mechanics-and-engineering}{Computer Methods in Applied Mechanics and Engineering}
\item \href{https://www.math.ualberta.ca/ijnam/}{International Journal of                                                 Numerical Analysis \& Modeling}
\item \href{https://onlinelibrary.wiley.com/journal/10970207}{International Journal for Numerical Methods in Engineering}
\item \href{https://onlinelibrary.wiley.com/journal/10982426}{Numerical Methods for Partial Differential Equations}
\item \href{https://www.tandfonline.com/journals/goms20}{Optimization Methods and Software}
\item \href{https://www.siam.org/publications/siam-journals/siam-journal-on-scientific-computing/}{SIAM Journal on Scientific Computing}

Already tried.  Editor pointed out references \cite{BraessCarstensenHoppe2007}, \cite{CarstensenHu2015}, \cite{Dorfler1996}.
\end{itemize}

\subsection*{Proof-focussed journals}

\begin{itemize}
\item \href{https://link.springer.com/journal/211}{Numerische Mathematik}
\item \href{https://www.siam.org/publications/siam-journals/siam-journal-on-numerical-analysis/}{SIAM Journal on Numerical Analysis}
%\item \href{}{}
\end{itemize}

{\small
\bibliographystyle{siamplain}
\bibliography{ab}
}
\end{document}
